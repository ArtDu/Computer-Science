\documentclass[a4paper,11pt]{book}
\usepackage[utf8]{inputenc} %кодировка текстового файла
\usepackage[english,russian]{babel} %локализация
\usepackage{indentfirst} %отступы и красные строки
\usepackage{geometry}
\usepackage{fancyhdr}
\usepackage{textcomp,latexsym,amsmath,amsfonts}%математические символы
\usepackage{graphics}%графика
\geometry{left=4.5cm}
\geometry{right=4.5cm}
\geometry{top=3.5cm}
\geometry{bottom=3cm}
\pagestyle{fancy}
\begin{document}
\fancyhead[L]{\small{\textup 146}}
\fancyhead[C]{\scriptsize{\textup{ДИФЕРЕНИЦАЛЬНЫЕ УРАВНЕНИЯ ВЫСШИХ ПОРЯДКОВ}}}
\fancyhead[R]{\small{\textup{[гл. \scriptsize{IV} }}}
\fancyfoot{} 
\renewcommand{\headrulewidth}{0pt}

\begin{text}
	
	\setlength{\parindent}{2em}
	\setlength{\parskip}{-0.4cm}

	\textbf{3.} {Уравнение вида:} 
	
	\textsl{$$F(y^{(n)},y^{(n-2)})=0.\eqno(21)$$} 
	 
	\noindent также и\hspace{0.5mm}н\hspace{0.5mm}т\hspace{0.5mm}е\hspace{0.5mm}г\hspace{0.5mm}р\hspace{0.5mm}и\hspace{0.5mm}р\hspace{0.5mm}у\hspace{0.5mm}ю\hspace{0.5mm}т\hspace{0.5mm}с\hspace{0.5mm}я\ в\ к\hspace{0.5mm}в\hspace{0.5mm}а\hspace{0.5mm}д\hspace{0.5mm}р\hspace{0.5mm}а\hspace{0.5mm}т\hspace{0.5mm}у\hspace{0.5mm}р\hspace{0.5mm}а\hspace{0.5mm}х. Введение нового пере-менного z = $y^{(n-2)}$ приводит уравнение (21) к уравнению второго порядка:

	\textsl{$$F(z^{(n)},z)=0.\eqno(22)$$} 

	\noindent Если уравнение (22) разрешено относительно $z^n$, т.е. имеет вид:

	\textsl{$$z^{n}=f(z),\eqno(21')$$} 

	\noindent то один из методов его интеграции таков: умножим обе части на $2z`$, получаем:

	\textsl{$$2z'z^n=2f(z)dz,$$} 

	\noindent или в диференциалах:

	\textsl{$$d(z'{}^{2})=2f(z)dz,$$} 	

	\noindent откуда

	
	  \textsl{$$z'{}^{2}=2 \mbox{\small$\int$} f(z)dz+C_1.$$} 
	
		
	\noindent Последнее уравнение можно разрешить относительно производной и разделить переменные:

	
	\large{\textsl{$$\frac{dz}{\sqrt{2\int{f(z)dz+C_1}}}=dx;$$} }\setlength{\parskip}{-0.1cm}
	
	
	\normalsize\noindent отсюда находим общий интеграл уравнения (22'):
	
	
	\textsl\mbox{\large{$$\int{\frac{dz}{\sqrt{2\int{f(z)dz+C_1}}}=x+C_2.}$$} }


	\normalsize\noindent Этот интеграл, по замене z на $y^{(y-2)}$, получает вид:\setlength{\parskip}{-0.2cm}

	\textsl{$$\Phi(y^{(n-2)},x,C_1,C_2)=0,$$}

	\noindent т. е. уравнение вида $(17');$ оно интегрируется, как мы уже знаем, квадратурами, причем эта интеграция ведет еще $n-2$ произвольных постоянных, и мы получим общее решение уравнения $(22').$\\
	\setlength{\parindent}{1.8em}
	\indent Если уравнение (21) дано в неразрешенном относительно $y^{(n)}$ виде и известно его параметрическое представление
	\setlength{\parindent}{2em}

	\textsl{$$y^{(n)}=\phi(t),\quad y^{(n-2)}=\psi(t), \eqno(21')$$}

	\noindent то интеграция совершается следующим образом. Мы имеем два равенства:

	\textsl{$$dy^{(n-1)}=y^{(n)}dx,\quad dy^{(n-2)}=y^{(n-1)}dx, $$}

	\noindent связывающих две неизвестные функции от t, именно x и y; исключая деление dx, получаем диференциальное уравнение для $y^{(n-1)}:$

	\textsl{$$y^{(n-1)}dy^{(n-1)}dx=y^{(n)}dy^{(n-2)}, $$}

\end{text}


\newpage
\fancyhead[L]{\scriptsize{\textup \textsc{\textsection2]}}}
\fancyhead[C]{\scriptsize{\textup{ТИПЫ УРАВНЕНИЙ} \large{n}\scriptsize-ГО ПОРЯДКА, РАЗРЕШАЕМЫЕ В КВАДРАТУРАХ }}
\fancyhead[R]{\scriptsize{\textup{147}}}
\fancyfoot{} 





\begin{text}
	

	\setlength{\parindent}{0cm}
	\setlength{\parskip}{-0.2cm}



	\noindent или в силу уравнений $(21')$

	\setlength{\parindent}{2em}

	\textsl{$$y^{(n-1)}dy^{(n-1)}=\varphi(t)\psi'(t)dt, $$}

	\noindent откуда квадратурой находим $(y^{(n-1)})^2;$ далее получим:

	\textsl{$$y^{(n-1)}=\sqrt{2\mbox{\small$\int{\varphi(t)}$}\psi'(t)dt+C.} $$}

	\noindent Имея параметрическое представление $y^{(n-1)}$ и $y^{(n-2)}$, мы свели задачу к типу $(20'')$, рассмотренному в разделе 2 настоящего параграфа. Дальнейшие квадратуры введут n---1 новых произвольных постоянных.\\

	\textit{Пример 5.} \large{$a^{2}\frac{d^{4}y}{dx^{4}}=\frac{d^{2}y}{dx^{2}}.$} \normalsize Полагая $y^n=z$, приходим к уравнению: $a^2z^n=z$, умножим обе части на $2z'$:

	\textsl{$$2a^2z'z^n=2zz'\quad \text{\textup{или}} \quad 2^2z'dz'=2z\ dz.$$}

	\noindent Интегрируя, находим:

	\textsl{$$a^2z^{'2}=z^2+C_1,$$}

	\noindent откуда

	\textsl{$$\frac{dz}{\sqrt{z^2+C_1}}=\frac{dx}{a}.$$}

	\noindent Вторая интеграция дает:

	\textsl{$$ln (z+\sqrt{z^2+C_1}) = \frac{x}{a}+lnC_2$$}

	\noindent или

	\textsl{$$z+\sqrt{z^2+C_1}=C_2e^{\frac{x}{a}},$$}

	\noindent Чтобы разрешить последнее уравнение относительно z, выгодно поступить следующем образом: делим 1 на обе части последнего равенства:

	\textsl{$$\frac{1}{z+\sqrt{z^2+C_1}}=\frac{1}{C_2}e^{-\frac{x}{a}},$$}

	\noindent в левой части освобождаемся от иррациональности в знаменателе, затем умножаем обе части на ---$C_1$:

	\textsl{$$z-\sqrt{z^2+C_1}=\frac{C_1}{C_2}e^{-\frac{x}{a}}.$$}

	\noindent Складывая это уравнение с исходным и деля на 2, получаем:

	\textsl{$$z=\frac{C_2}{2}e^{\frac{x}{a}}-\frac{C_1}{2C_2}e^{-\frac{x}{a}}.$$}	

	\noindent Подставляя вместо z его значение $y^n$ и интегрируя два раза, находим:

	\textsl{$$y=Ae^{\frac{x}{a}}+be^{-\frac{x}{a}}+Cx+D.$$}	

	\noindent где A, B, C, D --- произвольные постоянные.

\end{text}

\end{document}